\documentclass[raggedright, 11pt]{tufte-style-article}

\title{The \texttt{tufte-style-article} class}
\author{Sylvain Kern}


\begin{document}
	
\maketitle	

\noindent
\texttt{tufte-style-article}\margintext{Edward Tufte is a statistician, computer scientist and professor at Yale University. His personal website: \url{edwardtufte.com}} is a \LaTeX{} class with a design similar to Edward Tufte's works. His designs are known for their simplicty, legibleness, extensive use of sidenotes in a wide dedicated margin and tight graphic integration in the text. This class is however not a rigourous copy of Tufte's works, it is more of an inspiration. This also includes design features from \textit{The Elements of Typographic Style}.\marginnote{\textls{\scshape robert bringhurst}, \textit{The Elements of Typographic Style}, 1999.}

This documentation gives a glimpse of what the class looks like, while explaining how to install and use it. I tried to make it as complete as I could; some parts can still be unexplained or unclear. I tested it on several \LaTeX{} distributions, but it can still spit unexpected errors. Feel free to ask me for information or report a malfunction if you encounter one!\margintext{My email: \href{mailto:sylvain.kern98@gmail.com}{\texttt{sylvain.kern98\\@gmail.com}}}

I am aware that numerous Tufte-based classes exist around the Internet, I just wanted to create my own to really feel and internalize this design grammar. Eventually, this is just my view of what I find well-presented and eye-pleasing in a document and not prescriptive rules and guidelines. Everybody can feel free to modify, customize, and contribute to this class\margintext{I give a few information for contributors on section \ref{}}.



\tableofcontents

\newpage

\section{Installation}

\section{Presentation and Usage}

\subsection{The big margin}

\subsection{Figures, tables and stuff}

\subsection{Paragraph behavior}

\subsection{Code}

\begin{codebox}{c}
int main(int argc, char * argv[]) {
	return 0;	
}
\end{codebox}

\begin{codeboxnonos}{c}
int main(int argc, char * argv[]) {
	return 0;	
}
\end{codeboxnonos}

\begin{nextcodebox}{c}
int main(int argc, char * argv[]) {
	return 0;	
}
\end{nextcodebox}

Texte \inlinecode{python}{def f(x,y,z):} bonsoir

\section{Adaptation and customization}

\end{document}